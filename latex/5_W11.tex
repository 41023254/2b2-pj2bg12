\chapter{W11}


\section{41023234}
自評分數:65

這次四人分組 我主要把  \(組別亂數生成\)  完成&把  \(感測器\)\(場景\)\(記分板\)  生成並運轉出來


\section{41023247}
自評分數:62分
在這次的分組作業,學到了如何解決合併時所產生的問題,同時,在本次的模擬中提供了場地,也更了解如何將cad圖檔轉檔進CoppeliaSim。

而在這次的作業裡,我們發現到球門只會感應到機器人而不是球,一開始對於這個問題沒有任何想法解決,,直到詢問班上的別組同學,才知道要將機器人與球的物理性質更改,這才解決了這個煩人的問題。
\section{41023251}

自評分數:60

在4人協作的pj2開始時主要負責研究如何操作pull request,擬出一套操作步驟置於組內H1頁面。

同時維護網站時期主要研究如何多人同時維護網站,常與54號同學於放學後留下研究。

因為W9內H2頁面標題與W11重複,導致讀取content資料夾內html檔案時會有兩個相同檔名。因此將W9內之H2頁面重新命名。


\section{41023254}

我在這次四人協同的課程中主要負責程式的工作,在大家pull requests發生衝突時也會解決衝突,解決衝突的方法來自和51號同學放學後一次又一次的同時上傳測試。

自評:60