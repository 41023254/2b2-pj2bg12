\chapter{W11}


\section{41023234}

今天我們測試了雙聯機bubblerod,和建立4人網頁,之前一直只能一台控制沒辦法兩人聯機的問題終於有範例可以解決了,現在有對聯機有大概理解了,同時也感謝同組組員幫忙,讓我有比較理解。

\section{41023247}
自評分數:62分
在這次的分組作業,學到了如何解決合併時所產生的問題,同時,在本次的模擬中提供了場地,也更了解如何將cad圖檔轉檔進CoppeliaSim。

而在這次的作業裡,我們發現到球門只會感應到機器人而不是球,一開始對於這個問題沒有任何想法解決,,直到詢問班上的別組同學,才知道要將機器人與球的物理性質更改,這才解決了這個煩人的問題。
\section{41023251}

自評分數:60

在4人協作的pj2開始時主要負責研究如何操作pull request,擬出一套操作步驟置於組內H1頁面。

同時維護網站時期主要研究如何多人同時維護網站,常與54號同學於放學後留下研究。

因為W9內H2頁面標題與W11重複,導致讀取content資料夾內html檔案時會有兩個相同檔名。因此將W9內之H2頁面重新命名。


\section{41023254}

我在這次四人協同的課程中主要負責程式的工作,在大家pull requests發生衝突時也會解決衝突,解決衝突的方法來自和51號同學放學後一次又一次的同時上傳測試。

自評:60