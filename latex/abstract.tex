\renewcommand{\baselinestretch}{1.5} %設定行距
\pagenumbering{roman} %設定頁數為羅馬數字
\clearpage  %設定頁數開始編譯
\sectionef
\addcontentsline{toc}{chapter}{摘~~~要} %將摘要加入目錄

\begin{center}
\LARGE\textbf{摘~~要}\\
\end{center}

\begin{flushleft}

\fontsize{14pt}{20pt}\sectionef\hspace{12pt}\quad 此專題是運用bubblerob,將其導入 CoppeliaSim 模擬環境並給予對應設置,將其程式系統化並運用 zmq Remote api進行控制,找到適合此場景的程式寫法後,再到 CoppeliaSim 模擬環境中進行測試程式和場景在多人連線中的可行性。並嘗試透過架設伺服器將 CoppeliaSim 影像串流到網頁供使用者觀看或操控。


\end{flushleft}

\begin{center}
\fontsize{14pt}{20pt}\selectfont 關鍵字: 影像串流\sectionef bubblerob、zmq Remote api、CoppeliaSim
\end{center}
